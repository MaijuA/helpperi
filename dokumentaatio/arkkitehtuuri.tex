\documentclass[a4paper,12pt, titlepage]{article}
\usepackage[finnish]{babel} %suomenkielinen tavutus
\usepackage[T1]{fontenc} %skanditavutus
\usepackage[utf8]{inputenc}        	% skandit utf-8 koodauksella
%\usepackage[ansinew]{inputenc}        	% skandit utf-8 koodauksella, kokeile tata, jos utf-8 ylla ei toimi.

\usepackage{graphicx}

\linespread{1.24} %rivivali 1.5
\sloppy % Vahentaa tavutuksen tarvetta, "leventamalla" rivin keskella olevia valilyönteja.


\title{Helpperin arkkitehtuuri}
\author{ \\[1cm] Ohjelmiston arkkitehtuuridokumentaatio \\ Helsingin yliopisto}
\date{Toukokuu 2016}

\begin{document}

\maketitle

\tableofcontents

\section{Dokumentin tarkoitus}

\section{Järjestelmän yleiskuvaus}

\section{Relaatiotietokantakaavio}

\section{Järjestelmän yleisrakenne}

\section{Käyttöliittymä ja järjestelmän komponentit}

\section{Liitteet}

\end{document}
